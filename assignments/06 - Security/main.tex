\documentclass[11pt]{article}
\usepackage[utf8]{inputenc}


\newif\ifbr
\newif\ifen
\newif\iffr

% Output Language
%\brtrue \usepackage[brazilian]{babel}
\entrue \usepackage[english]{babel}
%\frtrue \usepackage[french]{babel}

\usepackage[parfill]{parskip}


\ifbr
\title{Security}
\author{
	Pedro Penna$^{1,2}$, Henrique Freitas$^{1}$,\\
	Márcio Castro$^{2}$, Jean-François Méhaut$^{3}$\\[0.3em]
	\small $^{1}$ Pontifícia Universidade Católica de Minas Gerais\\
	\small $^{2}$Universidade Federal de Santa Catarina\\
	\small $^{3}$ Universidade de Grenoble Alpes
}
\date{}
\else\ifen
\title{Security}
\author {
	Pedro Penna$^{1,2}$, Henrique Freitas$^{1}$,\\
	Márcio Castro$^{2}$, Jean-François Méhaut$^{3}$\\[0.3em]
	\small $^{1}$ Pontifical University of Minas Gerais (PUC Minas), Belo Horizonte, Brazil\\
	\small $^{2}$ Federal University of Santa Catarina (UFSC), Florianapolis, Brazil\\
	\small $^{3}$ Grenoble Alpes University (UGA), Polytech Grenoble, France}
\date {}
\fi\fi

\begin{document}

\maketitle

\begin{abstract}
\ifbr
	\noindent Além de prover uma abstração do hardware de maneira
	eficiente e robusta, o sistema operacional deve fazer isso de forma
	segura, certificando que nenhum usuário monopolize recursos, derrube
	o sistema e nem acesse informações sem autorização necessária, seja
	isso intencional ou não. Nesse projeto, você lidará com aspectos
	relacionados à segurança de um sistema operacional explorando
	vulnerabilidades presentes no Nanvix.
\else\ifen
	\noindent The operating system should abstract and multiplex the
	underlying hardware in an efficient and secure fashion. It should
	prevent users from monopolizing resources and overthrow the system
	itself, either intentionally or not. In this project, you shall deal
	with core concepts in security of an operating system by exploiting
	some vulnerabilities that exist in Nanvix.
\fi\fi

\end{abstract}

\ifbr
\subsubsection*{Fundamentação Teórica}

Sob o ponto de vista de segurança, as vulnerabilidades de um sistema
podem ser classificadas em três classes:

\begin{itemize}
	\item Confidencialidade
	\item Integridade
	\item Disponibilidade
\end{itemize}

Vulnerabilidades de confidencialidade e integridade dizem respeito ao
acesso e modificação não autorizado de dados, respectivamente. Por
exemplo, quando um usuário $A$ consegue visualizar documentos de um
outro usuário $B$ sem autorização prévia do último, tem-se uma
vulnerabilidade de confidencialidade. Por outro lado, se o usuário $A$
conseguir alterar o conteúdo desses documentos, dizemos que houve uma
vulnerabilidade de integridade, uma vez que a informação foi adulterada
pelo usuário (malicioso) $A$, sem o consentimento do usuário $B$. Em
resumo, essas duas vulnerabilidades estão relacionadas aos dados de
usuários e são geralmente exploradas através de falhas de segurança
presentes no sistema de arquivos, apesar de também serem possíveis
através técnicas de violação de memória.

Já vulnerabilidades de disponibilidade estão relacionadas à baixa (ou
nenhuma) disponibilidade para computação do sistema. Por exemplo,
imagine um programa de usuário $C$ que faça uso intenso do disco rígido,
por exemplo um programa de \textit{backup}. Caso o sistema não efetue o
escalonamento de operações de disco entre os diferentes processos
presentes no sistema de maneira justa, o programa $C$ pode acabar por
dominar as filas de disco fazendo com que operações dos demais processos
sejam postergadas por um longo tempo. Nesse cenário fictício, os
processos do sistema estariam prejudicados quanto à operações de entrada
e saída no disco rígido, mas ainda conseguiriam realizar computação. Em
um cenário mais problemático, no entanto, um processo $D$ poderia vir a
derrubar todo o sistema, após invocar por exemplo invocando uma chamada
de sistema inválida e provocando um pânico no \textit{kernel}.

\else\ifen
\subsubsection*{Background}

The vulnerabilities of an operating system may be grouped in three
classes:

\begin{itemize}
	\item Confidentiality
	\item Integrity
	\item Availability
\end{itemize}

Confidentiality and integrity concern unauthorized access and
modification of data, respectively. For instance, when some user $A$
is able to open and read files of another user $B$ without previous
permission, there is a confidentiality vulnerability in the system. On
the other hand, if $A$ succeeds in changing the contents of this file,
there is an integrity vulnerability in the system. In summary, these two
vulnerabilities concern the user data and oftentimes are exploited via
security breaches in the file system.

The third vulnerability class regards the low (or none) availability of
some resources in the system. For instance, suppose that a user program
$C$ makes intense use of the hard disk. If the operating system does not
schedule operations in the hard disk in a fair way among the several
processes that are running, the $C$ program may end up hugging the hard
disk for itself and thus causing other processes to starve. In this
scenario, other processes still can use other resources, such as the
processor. However, in an even more chaotic situation some process $D$
could overthrow the whole operating system if it succeeded in invoking
some operation that would cause a kernel panic.

\fi\fi

\ifbr
\subsubsection*{Descrição do Projeto}

Nesse projeto, você deverá encontrar e explorar vulnerabilidades
existentes no Nanvix. Intencionalmente, para cada uma das classes de
vulnerabilidade discutidas, algumas falhas foram deixadas no sistema:

\begin{itemize}
	\item Confidencialidade: exposição de senhas e de dados em memória
	\item Integridade: escalação de privilégios 
	\item Disponibilidade: negação e indisponibilidade de serviço
\end{itemize}

Sua tarefa consiste em encontrar e explorar ao menos três das cinco
falhas apontadas. Para isso, use os conceitos e técnicas vistos em sala
de aula, e recorra à documentação do sistema. Antes de iniciar o
projeto, certifique-se que você tem a versão mais recente do Nanvix, com
a opção de multiusuário habilitada\footnote{Para habilitar suporte à
multiusuário altere o arquivo
\texttt{include/nanvix/config.h}}\footnote{Usuário: \texttt{noob},
senha: \texttt{noob}.}. Você deverá entregar um relatório descrevendo as
falhas encontradas, como elas foram exploradas e os códigos-fonte
usados.

\else\ifen

\subsection*{Assignment Description}

In this assignment you shall find and exploit security breaches in
Nanvix. Intentionally, for each one of the aforementioned vulnerability
classes, some flaws were left in the operating system:

\begin{itemize}
	\item Confidentiality: exposure of passwords and data in memory.
	\item Integrity: privilege escalation
	\item Availability: denial of service and unavailability of service
\end{itemize}

Your should find and exploit at least three vulnerabilities in the
system. To this end, you refer to both, the textbook and the background
theory that we covered in the course.

Before starting, make sure that you have have enabled the multi-user
version of Nanvix enabled. To this, you should change the header file
\texttt{config.h} in \texttt{include/nanvix}. The default user is \texttt{noob} and
the default password is \texttt{noob}.

In the end, you should write a report about the security breaches that
you found and how you exploited them.

\fi\fi

\ifbr

\subsubsection*{Recompensas}

Além das falhas indicadas, uma falha de \textit{buffer overflow} está
presente no sistema. O grupo que conseguir encontrá-la e explorará-la
será devidamente recompensado.

\fi

\end{document}
