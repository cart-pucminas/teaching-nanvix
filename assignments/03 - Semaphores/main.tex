\documentclass[11pt]{article}
\usepackage[utf8]{inputenc}

\newif\ifbr
\newif\ifen
\newif\iffr

% Output Language
%\brtrue \usepackage[brazilian]{babel}
%\entrue \usepackage[english]{babel}
\frtrue \usepackage[french]{babel}

\usepackage[parfill]{parskip}

% Identação de blocos.
\usepackage{scrextend}

% URLs.
\usepackage[hidelinks,colorlinks=true,urlcolor=blue]{hyperref}

\ifbr

\title{Semáforos}

\author{Pedro H. Penna, João Caram e Henrique C. Freitas\\[0.3em]
\small Pontifícia Universidade Católica de Minas Gerais}
\date{}

\begin{document}

\maketitle

\begin{abstract}

\noindent Mecanismos de sincronização inter-processos possibilitam que processos acessem recursos compartilhados de maneira segura. Nesse projeto você implementará a abstração de semáforos no \textit{kernel}.

\end{abstract}

\subsubsection*{Fundamentação Teórica}

Um semáforo é uma primitiva de sincronização inter-processos proposta por Edsger Dijkstra em 1965. Nesse mecanismo, uma variável especial protegida, do tipo abstrato semáforo, possibilita o acesso exclusivo a recursos compartilhados por meio de quatro operações básicas:

\begin{itemize}
	\item \texttt{create(n)}: cria um semáforo semáforo e o inicializa com o valor \texttt{n}.
	\item \texttt{down()}: testa o valor do semáforo. Se esse valor for maior que zero, ele é decrementado e o processo continua sua execução normalmente. Caso contrário, o processo é bloqueado.
	\item \texttt{up()}: testa se o valor do semáforo é zero. Em caso afirmativo e existir algum processo bloqueado no semáforo, o processo é desbloqueado. Caso contrário o valor do semáforo é incrementado.
	\item \texttt{destroy()}: destrói o semáforo.
\end{itemize}

As operações \texttt{down()} e \texttt{up()} são atômicas, isto é, enquanto um processo executa uma dessas operações, nenhum outro pode executar outra operação sobre o mesmo semáforo. Assim, o acesso exclusivo a um recurso compartilhado é garantido.

\subsubsection*{Descrição do Projeto}

O Nanvix não oferece nenhuma primitiva de sincronização inter-processos. Processos podem manipular regiões de memória compartilhada e arquivos, mas sem a garantia de que não irão atrapalhar a computação de outros processos que acessam os mesmos recursos compartilhados.

Para resolver esse problema, você deverá implementar abstração de semáforos em \textit{kernel}. Essas operações devem ser exportadas como chamadas de sistema e devem seguir as seguintes especificações.

\texttt{int semget(unsigned key)}

	\begin{addmargin}[2em]{0em}

	A função \texttt{semget} permite que um processo use um semáforo associado a uma chave (\texttt{key}).
	Caso nenhum semáforo esteja associado a \texttt{key}, um novo semáforo deve ser criado.

	Em caso de conclusão com êxito, a função deve retornar o identificador do semáforo associado a \texttt{key}.
	Em caso de erro, -1 deve ser retornado.
	
	\end{addmargin}

\texttt{int semctl(int semid, int cmd, int val);}

	\begin{addmargin}[2em]{0em}

	A função \texttt{semctl()} permite um série de operações de controle no semáforo identificado por \texttt{semid}.
	Essas operações são especificadas por \texttt{cmd} e podem ser:
	\texttt{GETVAL}, retorna o valor corrente do semáforo;
	\texttt{SETVAL}, define o valor do semáforo como sendo \texttt{val}; e
	\texttt{IPC\_RMID}, desvincula o semáforo do processo corrente e o destrói caso não esteja mais sendo usado.

	O valor de retorno da função depende de \texttt{cmd}. 
	Caso esse seja \texttt{GETVAL}, o valor corrente do semáforo é retornado.
	Nos demais casos de conclusão com êxito, 0 deve ser retornado.
	Em caso de erro -1 deve ser retornado retornado.

	\end{addmargin}
	
\texttt{int semop(int semid, int op);}

	\begin{addmargin}[2em]{0em}

	A função \texttt{semop} permite operações atômicas de incremento/decremento no semáforo identificado por \texttt{semid}.	
	Um valor negativo para \texttt{op} especifica um operação \texttt{down()} e um valor positivo uma operação \texttt{up()}.
	
	A função deve retornar 0 em uma conclusão com êxito, ou então -1 em caso de erro.
	
	\end{addmargin}
	
Junto com o projeto você deverá entregar uma pequena descrição sobre a solução implementada. Programas-exemplo serão fornecidos para que você teste sua implementação.

\subsubsection*{Por Onde Começar?}

Nesse projeto os seguintes arquivos serão relevantes:

\begin{itemize}
    \item \texttt{include/nanvix/syscall.h}: IDs das chamadas de sistema.
    \item \texttt{kernel/pm/sleep.c}: implementação das funções \texttt{sleep()} e \texttt{wakeup()}.
    \item \texttt{kernel/sys/syscalls.c}: tabela de chamadas de sistema.
    \item \texttt{include/sys/sem.h}: interface de usuário para manipulação de semáforos.
    \item \texttt{lib/libc/sys/sem/semget.c}: função de biblioteca \texttt{semget()}.
    \item \texttt{lib/libc/sys/sem/semctl.c}: função de biblioteca \texttt{semctl()}.
    \item \texttt{lib/libc/sys/sem/semop.c}: função de biblioteca \texttt{semop()}.
\end{itemize}

No entanto, você também deverá criar alguns outros arquivos:

\begin{itemize}
    \item \texttt{kernel/sys/sem.c}: implementação da abstração de semáforos.
    \item \texttt{kernel/sys/sem/semget.c}: chamada de sistema \texttt{sys\_semget()}.
    \item \texttt{kernel/sys/sem/semctl.c}: chamada de sistema \texttt{sys\_semctl()}.
    \item \texttt{kernel/sys/sem/semop.c}: chamada de sistema \texttt{sys\_semop()}.
\end{itemize}

\subsubsection*{Indo Além}

especificações para manipulação de semáforos nesse projeto consistem em uma simplificação da versão estabelecida pelo padrão Posix, disponível em \url{pubs.opengroup.org/stage7tc1/basedefs/sys\_sem.h.html} 

Caso você deseje implementar uma interface mais robusta, que esteja mais conforme o Posix, você será recompensado pelo seu esforço. Procure o professor da disciplina para discutir o assunto em maiores detalhes. Além disso, o grupo que possuir a implementação mais robusta será convidado a submeter um \textit{pull} no repositório de desenvolvimento do Nanvix.


\else\iffr
\title{Sémaphores}

\author{Pedro H. Penna, João Caram, Henrique C. Freitas\\[0.3em]
  \small Pontifícia Universidade Católica de Minas Gerais (PUC Minas)\\[0.8em]
  Jean-François Méhaut\\[0.3em]
  \small Université Grenoble Alpes (UGA)
}
\date{}

\begin{document}

\maketitle

\begin{abstract}

  \noindent Les mécanismes de synchronisation, comme des sémaphores ou
  des moniteurs permettent aux processus d'accéder à des ressources
  partagées de manière coordonnée et sécurisée. Dans ce projet, vous allez
  implémenter les sémaphores dans le \textit {kernel} (noyau) du
  système Nanvix.
  
\end{abstract}

\subsubsection*{Fondements théoriques}

Un sémaphore est objet qui va permettre la synchronisation de
processus proposée par Edsger Dijkstra en 1965. Le sémaphore est
composé d'une variable entière (sa valeur) et d'une file d'attente
(les processus bloqués sur ce sémaphore). Le sémaphore permet un accès
coordonné et synchronisé aux ressources partagées avec quatre
opérations de base:

\begin{itemize}
\item \texttt{create(n)}: créé un sémaphore de sémaphore et l'initialise avec la valeur \texttt {n}.

\item \texttt{down()}: teste la valeur du sémphore. Si cette valeur est
  supérieure à zéro, elle est décrémentée et le processus appelant continue
  son exécution normalement. Sinon, le processus est bloqué sur la file
  d'attente associée au sémaphore.

\item \texttt{up()}: teste si la valeur du sémaphore est nulle. Si c'est
  le cas, et qu'il y a un processus bloqué dans la file d'attente du sémaphore,
  ce processus est débloqué. Sinon, la valeur du sémaphore est incrémentée.

\item \texttt{destroy()}: détruit le sémaphore.
\end{itemize}

Les opérations \texttt {down} et \texttt {up} sont atomiques,
c'est-à-dire que lorsqu'un processus exécute l'une de ces opérations,
aucune autre opération ne peut être exécutée sur le même sémaphore.
Ainsi, l'accès exclusif à une ressource partagée peut être garanti.

\subsubsection*{Description du projet}

Dans la version initiale du système, Nanvix n'offre aucune primitive
de synchronisation inter-processus. Les processus peuvent avoir à
gérer des régions et des fichiers en mémoire partagée, mais sans avoir
l'assurance qu'ils ne perturberont pas le calcul des autres processus
qui accèdent aux mêmes ressources partagées.

Pour résoudre ce problème, vous devez implémenter l'abstraction de
sémaphore dans le \textit {kernel} (noyau) du système d'exploitation
Nanvix. Ces opérations doivent être exportées en tant qu'appels
système (\textit{system call}) et doivent respecter les spécifications
suivantes.

\texttt{int semget(unsigned key)}

	\begin{addmargin}[2em]{0em}
          La fonction \texttt {semget} permet à un processus d'utiliser
          un sémaphore associé à une clé (\ texttt {key}). Si aucun sémaphore
          n'est associé à \texttt {key}, un nouveau sémaphore doit être créé.

          En cas de réussite, la fonction retournera l'identificateur
          du sémaphore associé à la \texttt {clé}.  En cas d'erreur,
          $-1$ doit être retourné.
	\end{addmargin}

\texttt{int semctl(int semid, int cmd, int val);}

\begin{addmargin}[2em]{0em}

  La fonction \texttt {semctl ()} permet une série d'opérations de contrôle
  sur le sémaphore identifié par \texttt {semid}. Ces opérations sont
  spécifiées par \texttt {cmd} et peuvent être:
\begin{itemize}
  \item \texttt{GETVAL}, renvoie la valeur actuelle du sémaphore;
  \item \texttt {SETVAL}, définit la valeur du sémaphore sur \texttt {val};
  \item \texttt {IPC\_RMID}, supprime le sémaphore et le détruit s'il n'est
    plus utilisé.
\end{itemize}

La valeur de retour de la fonction dépend de \texttt {cmd}.
Si c'est \texttt {GETVAL}, la valeur actuelle du sémaphore est retournée.
Dans tous les autres cas d'achèvement réussi, $0$ doit être retourné.
En cas d'erreur, $-1$ doit être retourné retourné.

\end{addmargin}
	
\texttt{int semop(int semid, int op);}

\begin{addmargin}[2em]{0em}

  La fonction \texttt {semop} permet d'effectuer des opérations atomiques
  incrémentant ou décrémentant la variable associée au sémaphore identifié
  par \texttt {semid}.
  
  Une valeur négative pour \texttt {op} spécifie l'opération \texttt
  {downw ()} et une valeur positive l'opération \texttt {up ()}.

La fonction retourne $0$ en cas de réussite, ou $-1$ en cas d'erreur.
	
\end{addmargin}

Avec le projet, vous devez fournir une brève description de la solution mise
en {\oe}uvre. Vous implémenterez un programme utilisateur avec des
processus \textit{producteurs} et des processus \textit{consommateurs}.


\subsubsection*{Par où commencer?}

Pour ce projet, nous vous suggèrons d'étudier le contenu
des fichiers suivants:


\begin{itemize}
    \item \texttt{include/nanvix/syscall.h}: Identificateurs (ID) des appels système.
    \item \texttt{kernel/pm/sleep.c}: Implémentations des fonctions \texttt{sleep()} et \texttt{wakeup()}.
    \item \texttt{kernel/sys/syscalls.c}: Table des appels système.
    \item \texttt{include/sys/sem.h}: Interface utilisateur pour la
      manipulation des sémaphores.
    \item \texttt{lib/libc/sys/sem/semget.c}: fonction de la
      bibliothèque \texttt{semget()}.
    \item \texttt{lib/libc/sys/sem/semctl.c}: fonction de la biblothèque \texttt{semctl()}.
    \item \texttt{lib/libc/sys/sem/semop.c}: fonction de la bibliothèque \texttt{semop()}.
\end{itemize}

Pour ce projet, vous aller devoir créer d'autres fichiers:

\begin{itemize}
    \item \texttt{kernel/sys/sem.c}: Implémentation de l'abstraction sémaphore.
    \item \texttt{kernel/sys/sem/semget.c}: Appel système \texttt{sys\_semget()}.
    \item \texttt{kernel/sys/sem/semctl.c}: Appel système \texttt{sys\_semctl()}.
    \item \texttt{kernel/sys/sem/semop.c}: Appel système \texttt{sys\_semop()}.
\end{itemize}

\subsubsection*{Aller au delà}

Les spécifications pour le traitement des sémaphores dans ce projet
sont une version simplifiée de la version établie par la norme
IEEE Posix, disponible sur \url {pubs.opengroup.org/stage7tc1/basedefs/sys\_sem.h.html}

Si vous voulez implémenter une interface plus robuste et plus conforme
à Posix, vous serez récompensés pour vos efforts. Pour plus détail, vous pouvez discuter
avec les enseignants.

\end{document}

\fi\fi
\end{document}
