\documentclass[11pt]{article}
\usepackage[utf8]{inputenc}
\usepackage[brazilian]{babel}

\usepackage[parfill]{parskip}

\title{Compilando e Testando o Nanvix}

\author{Pedro H. Penna, João Caram e Henrique C. Freitas\\[0.3em]
\small Pontifícia Universidade Católica de Minas Gerais}
\date{}

\hyphenation{tool-chain}

\begin{document}

\maketitle

\begin{abstract}
\noindent Antes de começar a \textit{hackear} o Nanvix, é fundamental que você se familiarize com o modo como os arquivos do projeto estão organizados, com quais ferramentas de desenvolvimento você vai lidar e com o processo de compilação. Nesse primeiro projeto, você vai trabalhar em todas essas tarefas e, ao final, estará pronto para começar o trabalho duro.
\end{abstract}

\subsubsection*{A Estrutura do Projeto}

Por tratar-se de um projeto ligeiramente grande e complexo, o Nanvix é organizado em uma hierarquia de diretórios, que é detalhada a seguir:

\begin{itemize}
    \item \texttt{bin} conterá o binário do \textit{kernel} e utilitários, depois de terem sido compilados.
    \item \texttt{doc} contém toda a documentação do Nanvix, que inclui manuais do sistema, de bibliotecas e utilitários; orientações gerais para desenvolvimento; e documentação de APIs.
    \item \texttt{doxygen} contém arquivos de configuração da ferramenta Doxygen, que gera documentação das APIs do Nanvix diretamente do código fonte.
    \item \texttt{include} contém os arquivos-cabeçalhos de escopo global, tanto de sistema quanto de biblioteca.
    \item \texttt{lib} conterá todas as bibliotecas estáticas e dinâmicas, depois de terem sido compiladas.
    \item \texttt{src} contém o código fonte do Nanvix.
    \item \texttt{tools} contém todas as ferramentas e \textit{scripts} necessários para compilar o Nanvix.
\end{itemize}

\newpage

\subsubsection*{Ferramentas e Ambiente de Desenvolvimento}

As ferramentas utilizadas no desenvolvimento do Nanvix, como no desenvolvimento de qualquer outro sistema operacional, se diferem das utilizadas no desenvolvimento da maioria dos outros tipos de software.

Em primeiro lugar, as ferramentas de compilação devem ser compatíveis com a plataforma alvo. Por exemplo, o Nanvix foi projetado para a plataforma x86, portanto as ferramentas utilizadas para compilar o sistema devem ser capazes de gerar código de máquina para essa plataforma. Em segundo lugar, quando trabalha-se em nível de \textit{kernel}, o ambiente de desenvolvimento não provê quaisquer tipo de bibliotecas padrões. Em terceiro lugar, para testar o sistema deve-se utilizar uma máquina dedicada, seja ela real ou virtual. Finalmente, em quarto lugar, as ferramentas de \textit{debugging} disponíveis são restritas, mas extremamente poderosas.

Para o desenvolvimento do Nanvix, você utilizará duas ferramentas: a \textit{toolchain} GCC-x86, uma coletânea de utilitários que inclui compilador, \textit{assembler} e \textit{linker}; e o Bochs, um emulador para plataforma x86 que possui um poderoso \textit{debugger} embutido. Para instalar essas ferramentas de forma automática, execute\footnote{\label{note1}Talvez você precise de uma autenticação privilegiada para executar essa ação.}, do diretório raiz do projeto, os \textit{scripts} \texttt{setup-bochs.sh} e \texttt{setup-toolchain.sh}, ambos localizados no diretório \texttt{tools/dev}.

\subsubsection*{Compilando e Testando}

Uma vez que as ferramentas de desenvolvimento necessárias para compilar o Nanvix foram devidamente instaladas, compilar o sistema torna-se uma tarefa simples. Do diretório raiz do projeto, invoque o utilitário \texttt{make} com o argumento \texttt{nanvix}. Terminado esse processo, todos os binários devem ter sido criados com sucesso.

Agora, para gerar uma imagem do sistema, re-invoque\textsuperscript{\ref{note1}} o utilitário \texttt{make} do diretório raiz, mas dessa vez com o argumento \texttt{image}. Isso fará com que o arquivo \texttt{nanvix.img} seja criado no diretório raiz. Esse arquivo é a imagem do sistema e será usado no processo de \textit{boot}, a seguir.

Finalmente, inicie o Nanvix no Bochs, dando \textit{boot} pela imagem de sistema gerada. Para fazer isso de forma automática, execute\textsuperscript{\ref{note1}}, a partir do diretório raiz, o \textit{script} \texttt{run.sh} localizado no diretório \texttt{tools/run}.

\end{document}
