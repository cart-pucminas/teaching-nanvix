\documentclass[11pt]{article}
\usepackage[utf8]{inputenc}
\usepackage[brazilian]{babel}


\usepackage[parfill]{parskip}

\title{\textit{Prefetching}}

\author{Pedro H. Penna, João Caram e Henrique C. Freitas\\[0.3em]
\small Pontifícia Universidade Católica de Minas Gerais}
\date{}


\begin{document}

\maketitle

\begin{abstract}

\noindent O subsistema de E/S deve prover ao usuário acesso eficiente aos dispositivos de E/S. Se isso não ocorrer, o usuário certamente ficará frustrado com a sua experiência virtual. Nesse projeto você implementará uma técnica que significativamente potencializa o desempenho de dispositivos de armazenamento em massa, o \textit{prefetching}.

\end{abstract}

\subsubsection*{Fundamentação Teórica}

Dentre os diferentes dispositivos de E/S existentes, podemos citar discos magnéticos como fundamentais, por servirem como principal local para persistência de dados de usuários e ao próprio sistema operacional como meio para possibilitar a técnica de memória virtual.

No entanto, esses dispositivos possuem o inconveniente de apresentarem baixas taxas de leitura e escrita, tornando operações lentas. Por exemplo, uma única requisição de leitura pode levar até 30 ms, em contraste a 0.1 ms para a mesma operação em memória. O principal motivo disso, se deve ao fato de discos magnéticos serem dispositivos escensialmente mecânicos.

Por isso, para minimizar os efeitos negativos de desempenho, técnicas, tais como agendamento de requisições, \textit{caching} e \textit{prefetching}, são usualmente aplicadas. Na técnica de agendamento, requisições de disco são agendadas de forma a minimizar o tempo de movimentação do braço mecânico do disco magnético. Na técnica de \textit{caching}, blocos mais usados do disco são mantidos em uma \textit{cache} em memória, de forma que requisições podem ser satisfeitas mais prontamente. Por fim, na técnica de \textit{prefetching}, dados são lidos antecipadamente do disco para a \textit{cache} do disco em memória, assim requisições de leitura podem também serem satisfeitas mais rapidamente.

\subsubsection*{Descrição do Projeto}

O subsistema de E/S do Nanvix adota a técnica de \textit{caching} para maximizar seu desempenho. Requisições entre processos de usuário e dispositivos de saída são atendidas diretamente de \textit{buffers} de blocos, em memória. O gerenciador de E/S mantém \textit{buffers} em uma \textit{cache} e utiliza um esquema LRU (\textit{least recently used}) para fazer a substituição de \textit{buffers} antigos por \textit{buffers} novos, quando necessário.

Apesar de simples, essa técnica confere um bom desempenho ao subsistema de E/S, tanto em requisições de leitura quanto de escrita. Para requisições de escrita especificamente, o uso de \textit{buffers} possibilita \textit{dealyed write}, reduzindo a banda média de transferência -- \textit{buffers} serão utilizados ao máximo antes de serem escritos de volta em disco. Para requisições de leitura e escrita, o uso da \textit{cache} possibilita que requisições sejam atendidas mais prontamente.

Ainda assim, o desempenho do subsistema de E/S do Nanvix ser melhorado ainda mais. Por exemplo, imagine um processo que realiza acesso sequencial a um arquivo, como um mp3 \textit{player}. Quando apenas a técnica de \textit{caching} é usada, toda vez que uma requisição de leitura de um novo bloco for feita, uma falta na \textit{cache} será gerada e o bloco deverá ser carregado diretamente do disco. Alternativamente, a técnica de \textit{prefetching} pode ser adotada para que blocos seguintes sejam carregados antecipadamente, aumentando a taxa de acertos na \textit{cache}. Assim, seu trabalho se resume a implementar essa funcionalidade no Nanvix.

\subsubsection*{Por Onde Começar?}

O código do gerenciador de E/S do Nanvix está no diretório \texttt{kernel/fs}, dividido em vários arquivos. No entanto, nesse projeto, você deve se atentar ao arquivo \texttt{buffer.c}, onde todas as funções para o gerenciamento da \textit{cache} de \textit{buffers} estão implementados. Em especial, estude a funação \texttt{bread()}, que realiza a leitura de um \textit{buffer} de bloco. Uma boa ideia seria utilizar essa função para criar a nova funcionalidade de \textit{prefetching}. Para testar sua estratégia de \textit{prefetching} use o utilitario \texttt{test}.

\subsubsection*{Recompensas}

O grupo que possuir a implementação mais robusta será convidado a submeter um \textit{pull} no repositório de desenvolvimento do Nanvix.

\end{document}
